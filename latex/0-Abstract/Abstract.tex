Die 3D Objekterkennung ist ein wichtiges Themengebiet der Mobilen Systeme und der autonomen mobilen Robotik geworden.
Ein populärer Ansatz um die Ähnlichkeit zwischen 3D Objekten zu bestimmen, sind globale Deskriptoren. Im Zuge
dieser Ausarbeitung für das Proseminar "`Mobile Systems Engineering"' wurde der wissenschaftlichen Artikel 
"`Histograms of Oriented Gradients for 3D Object Retrieval"' von Maximilian Scherer, Micheal Walter und Tobias Schreck 
ausgewählt und der darin beschriebene und entwickelte Deskriptor, im folgenden 3DHOG genannt, genauer vorgestellt.
In dieser Ausarbeitung wird zunächst der 2DHOG erläutert, welcher für die Objekterkennung in 2D Bildern eingesetzt wird und eine Grundlage
für den 3DHOG darstellt. Es folgt eine darauf aufbauende Erläuterung des 3DHOG, sowie ein Experiment, welches in \cite{scherer2010histograms}
durchgeführt wurde, um die Performanz des 3DHOG im Vergleich zu anderen Deskriptoren zu testen.