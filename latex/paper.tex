\documentclass[a4paper, 10pt, conference]{ieeeconf_de}


\overrideIEEEmargins

% See the \addtolength command later in the file to balance the column lengths
% on the last page of the document

\usepackage{graphicx}
\usepackage{amsmath}
\usepackage{amssymb}
\usepackage[latin1]{inputenc}
\usepackage[ngerman]{babel}


\title{\LARGE \bf
	Titel meiner Ausarbeitung
}

\author{Max Mustermann}


\begin{document}

\maketitle
\thispagestyle{empty}
\pagestyle{empty}


%%%%%%%%%%%%%%%%%%%%%%%%%%%%%%%%%%%%%%%%%%%%%%%%%%%%%%%%%%%%%%%%%%%%%%%%%%%%%%%%

\begin{abstract}

	

\end{abstract}

%%%%%%%%%%%%%%%%%%%%%%%%%%%%%%%%%%%%%%%%%%%%%%%%%%%%%%%%%%%%%%%%%%%%%%%%%%%%%%%%

\section{EINLEITUNG}


\section{HAUPTTEIL}

Einige Referenzen sind \cite{Hartley03}, \cite{Arulampalam02}, \cite{scherer2010histograms} , \cite{dalal2005histograms} und \cite{Kaestner05}.

\subsection{Meine erste Sektion} 
\tablename~\ref{table_example} zeigt etwas.

\begin{table}
\caption{Ein Beispiel für eine Tabelle.}
\label{table_example}
\begin{center}
\begin{tabular}{|c||c|}
\hline
One & Two\\
\hline
Three & Four\\
\hline
\end{tabular}
\end{center}
\end{table}


\subsection{Meine zweite Sektion}

\figurename~\ref{figurelabel} zeigt etwas.

   \begin{figure}[thpb]
      \centering
			%\includegraphics[width=\linewidth]{filename}
			Hier sollte ein Bild sein.
      \caption{Inductance of oscillation winding on amorphous
       magnetic core versus DC bias magnetic field}
      \label{figurelabel}
   \end{figure}

\section{DISKUSSION}

   

\addtolength{\textheight}{-12cm}  % This command serves to balance the column lengths
                                  % on the last page of the document manually. It shortens
                                  % the textheight of the last page by a suitable amount.
                                  % This command does not take effect until the next page
                                  % so it should come on the page before the last. Make
                                  % sure that you do not shorten the textheight too much.

%%%%%%%%%%%%%%%%%%%%%%%%%%%%%%%%%%%%%%%%%%%%%%%%%%%%%%%%%%%%%%%%%%%%%%%%%%%%%%%%

% References (BibTeX)
\bibliographystyle{plain}
\bibliography{paper}

\end{document}
